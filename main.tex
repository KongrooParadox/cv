%% Use the "normalphoto" option if you want a normal photo instead of cropped to a circle
% \documentclass[10pt,a4paper,withhyper,normalphoto]{altacv}

\documentclass[10pt,a4paper,withhyper]{altacv}
%% AltaCV uses the fontawesome5 and simpleicons packages.
%% See http://texdoc.net/pkg/fontawesome5 and http://texdoc.net/pkg/simpleicons for full list of symbols.

% Change the page layout if you need to
\geometry{left=1.25cm,right=1.25cm,top=1.25cm,bottom=1.25cm,columnsep=1.2cm}

% The paracol package lets you typeset columns of text in parallel
\usepackage{paracol}

% Change the font if you want to, depending on whether
% you're using pdflatex or xelatex/lualatex
% WHEN COMPILING WITH XELATEX PLEASE USE
% xelatex -shell-escape -output-driver="xdvipdfmx -z 0" sample.tex
\iftutex
  % If using xelatex or lualatex:
  \setmainfont{Roboto Slab}
  \setsansfont{Lato}
  \renewcommand{\familydefault}{\sfdefault}
\else
  % If using pdflatex:
  \usepackage[rm]{roboto}
  \usepackage[defaultsans]{lato}
  % \usepackage{sourcesanspro}
  \renewcommand{\familydefault}{\sfdefault}
\fi

% Change the colours if you want to
\definecolor{SlateGrey}{HTML}{2E2E2E}
\definecolor{LightGrey}{HTML}{666666}
\definecolor{DarkPastelRed}{HTML}{450808}
\definecolor{PastelRed}{HTML}{8F0D0D}
\definecolor{GoldenEarth}{HTML}{E7D192}
\colorlet{name}{black}
\colorlet{tagline}{PastelRed}
\colorlet{heading}{DarkPastelRed}
\colorlet{headingrule}{GoldenEarth}
\colorlet{subheading}{PastelRed}
\colorlet{accent}{PastelRed}
\colorlet{emphasis}{SlateGrey}
\colorlet{body}{LightGrey}

% Change some fonts, if necessary
\renewcommand{\namefont}{\Huge\rmfamily\bfseries}
\renewcommand{\personalinfofont}{\footnotesize}
\renewcommand{\cvsectionfont}{\LARGE\rmfamily\bfseries}
\renewcommand{\cvsubsectionfont}{\large\bfseries}


% Change the bullets for itemize and rating marker
% for \cvskill if you want to
\renewcommand{\cvItemMarker}{{\small\textbullet}}
\renewcommand{\cvRatingMarker}{\faCircle}
% ...and the markers for the date/location for \cvevent
% \renewcommand{\cvDateMarker}{\faCalendar*[regular]}
% \renewcommand{\cvLocationMarker}{\faMapMarker*}


% If your CV/résumé is in a language other than English,
% then you probably want to change these so that when you
% copy-paste from the PDF or run pdftotext, the location
% and date marker icons for \cvevent will paste as correct
% translations. For example Spanish:
% \renewcommand{\locationname}{Ubicación}
% \renewcommand{\datename}{Fecha}

%% Use (and optionally edit if necessary) this .tex if you
%% want an originally numerical reference style like IEEE
%% for your publication list
% \input{pubs-num}

\begin{document}
\name{Guillaume NANTY}
\tagline{Site Reliability Engineer}
\photoR{3.7cm}{picture}
% \photoL{3.2cm}{picture}

\personalinfo{%
  % Not all of these are required!
  \email{nanty.guillaume@proton.me}
  \phone{+33 626877533}
  % \mailaddress{269 Chemin de Cravailleux, 30126}
  \location{Tavel, FRANCE}
  % \homepage{kongrooparadox.github.io}
  % \twitter{@twitterhandle}
  % \xtwitter{@x-handle}
  \linkedin{guillaume_nanty_30_en}
  \github{KongrooParadox}
  % \orcid{0000-0000-0000-0000}
  %% You can add your own arbitrary detail with
  %% \printinfo{symbol}{detail}[optional hyperlink prefix]
  % \printinfo{\faPaw}{Hey ho!}[https://example.com/]

  %% Or you can declare your own field with
  %% \NewInfoFiled{fieldname}{symbol}[optional hyperlink prefix] and use it:
  % \NewInfoField{gitlab}{\faGitlab}[https://gitlab.com/]
  % \gitlab{KongrooParadox}
  %%
  %% For services and platforms like Mastodon where there isn't a
  %% straightforward relation between the user ID/nickname and the hyperlink,
  %% you can use \printinfo directly e.g.
  % \printinfo{\faMastodon}{@username@instace}[https://instance.url/@username]
  %% But if you absolutely want to create new dedicated info fields for
  %% such platforms, then use \NewInfoField* with a star:
  % \NewInfoField*{mastodon}{\faMastodon}
  % %% then you can use \mastodon, with TWO arguments where the 2nd argument is
  % %% the full hyperlink.
  % \mastodon{@username@instance}{https://instance.url/@username}
}

\makecvheader
%% Depending on your tastes, you may want to make fonts of itemize environments slightly smaller
% \AtBeginEnvironment{itemize}{\small}

%% Set the left/right column width ratio to 6:4.
\columnratio{0.6}

% Start a 2-column paracol. Both the left and right columns will automatically
% break across pages if things get too long.
\begin{paracol}{2}
\cvsection{Experience}

\cvevent{Freelance – Production Engineer}{FDJ United}{Jan 2024 -- Ongoing}{Vitrolles, France (13)}
I ensure the smooth operation of the FDJ Illiko digital games platform within a
team of 10 people.

\underline{Tasks :}
\begin{itemize}
\item Carry out Production Deployments (MEP) and Service Launches
\item Ensure the working order of the platform through observability
\item Contribute to the technical testing strategy
\item Implement orchestration of workloads with argo-workflows
\item Implement chaos engineering
\end{itemize}

\underline{Tech stack :}
\begin{itemize}
\item Kubernetes (Rancher), Helmfile, PostgreSQL, Kafka, Grafana, Gitlab, Argo-workflows, Litmuschaos
\end{itemize}

\divider

\cvevent{Lead Devops Engineer}{PACK Solutions}{Jan 2022 -- Dec 2023}{Les Angles, France (30)}
My job is to design and implement the infrastructure and CI for a complete rework of the companies' software product.
A cutting edge Web suite based on microservices architecture and driven by a BPMN engine.

\underline{Tasks :}
\begin{itemize}
\item Design and implementation of the CI/CD (Gitlab)
\item Setup \& admin of on-premises vanilla Kubernetes cluster
\end{itemize}
\underline{Tech stack:}
\begin{itemize}
\item Angular 14, Java 17, Maven, Quarkus, Kafka, Kogito, PostgreSQL, Keycloak, HashiCorp Vault
\end{itemize}

\divider

\cvevent{Release Manager}{PACK Solutions}{Feb 2019 -- Dec 2021}{Les Angles, France (30)}
Working in a small team of 2 people, I actively collaborate with both the development teams and the infrastructure department, working on the internally developed insurance management software.
\underline{Tasks :}
\begin{itemize}
\item Handle the configuration and deployment of all environments
\item Manage the version control sytems, the Continuous Integration tooling and releases
\item Implement SAST analysis of all projects with SonarQube, Angular CI/CD with Jenkins pipelines source code search and cross reference engine OpenGrok, centralized logging with Elastic Stack
\end{itemize}
\underline{Tech stack:}
\begin{itemize}
\item Angular, Java, Tomcat, Oracle, Flyway, Delphi, PHP, Kafka, SVN, Git
\end{itemize}

% \cvsection{Projects}
%
% \cvevent{Project 1}{Funding agency/institution}{}{}
% \begin{itemize}
% \item Details
% \end{itemize}
%
% \divider
%
% \cvevent{Project 2}{Funding agency/institution}{Project duration}{}
% A short abstract would also work.
%
% \medskip
%
% use ONLY \newpage if you want to force a page break for
% ONLY the current column
\newpage

%% Switch to the right column. This will now automatically move to the second
%% page if the content is too long.
\switchcolumn

\cvsection{About me}

\begin{quote}
Passionate about the DevOps philosophy, automation, Linux systems, and optimizing my workflow.

I thrive working under pressure, either independently or as part of a team.
\end{quote}

\cvsection{Strengths}

\cvtag{Hard-working}
\cvtag{Pragmatic}
\cvtag{Problem solving}
\cvtag{Self-learning}

\divider\smallskip

\cvtag{Active listening}
\cvtag{Empathetic}
\cvtag{Team player}
\cvtag{Communication skills}

% \divider\smallskip
%
% \cvtag{C++}
% \cvtag{Embedded Systems}\\
% \cvtag{Statistical Analysis}
%
\cvsection{Skills}

% Adapted from @Jake's answer from http://tex.stackexchange.com/a/82729/226
% \wheelchart{outer radius}{inner radius}{
% comma-separated list of value/text width/color/detail}
\wheelchart{1.5cm}{0.7cm}{%
  4/3em/accent!30/{Terraform, Ansible},
  6/3em/accent!40/Git,
  8/3em/accent!60/{CI/CD},
  4/4em/accent/Nix,
  5/2em/accent!20/{Python, Bash}
}

\cvsection{Languages}

\cvskill{English}{5}
\divider

\cvskill{French}{5}
\divider

\cvskill{Spanish}{4}

%% Yeah I didn't spend too much time making all the
%% spacing consistent... sorry. Use \smallskip, \medskip,
%% \bigskip, \vspace etc to make adjustments.
\medskip

\cvsection{Education}

\cvevent{Bachelor’s \& Master’s Degree in Computer Science and Networking}{University of Avignon, (France)}{Sept 2015 -- June 2019}{}
\divider

% \cvevent{ Degree in Computer Networking}{University of Avignon, (France)}{Sept 2015 -- June 2017}{}
% \divider
%
\cvevent{BTS SIO - 2 year post A levels course in web development}{Lycée Honoré d'Estienne d'Orves, Nice (France)}{Sept 2013 -- June 2015}{}

\end{paracol}


\end{document}
